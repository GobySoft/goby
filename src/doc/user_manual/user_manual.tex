% t. schneider

%!TEX TS-program = xelatex
%!TEX encoding = UTF-8 Unicode

\documentclass[11pt, letterpaper, oneside]{memoir}
\listfiles
\usepackage{fontspec} 

% FONTS
\defaultfontfeatures{Mapping=tex-text} % converts LaTeX specials (``quotes'' --- dashes etc.) to unicode
\setromanfont{Gentium}
\setverbatimfont{\normalfont\ttfamily\footnotesize}

% HEADINGS
\usepackage{afterpage}
\usepackage{pdflscape}
\usepackage{float}
\usepackage{graphicx} 
\usepackage{graphviz}
\graphicspath{{@RELATIVE_CMAKE_CURRENT_SOURCE_DIR@/../images/}}
\usepackage[normalem]{ulem} 
\usepackage{threeparttable}
\usepackage[cmex10]{amsmath}
\usepackage{color}
\usepackage[font=footnotesize]{subfig}
\definecolor{goby_orange}{RGB}{227,96,52}
\renewcommand{\colorchapnum}{\color{goby_orange}}
\renewcommand{\colorchaptitle}{\color{goby_orange}}
\definecolor{goby_aqua}{RGB}{28,159,203}
\definecolor{goby_aqua_darkened}{RGB}{14,77,94}

\setlength{\parindent}{0.0in}
\setlength{\parskip}{0.1in}

% PDF SETUP
\usepackage[bookmarks, colorlinks, pdftitle={Goby User Manual},pdfauthor={Goby Developers}]{hyperref} 
\hypersetup{linkcolor=goby_aqua_darkened,citecolor=goby_aqua_darkened,filecolor=black,urlcolor=goby_aqua_darkened} 
\newcommand{\xmltag}[1]{\texttt{$<$#1$>$}}

\setsecnumdepth{subsection}

% GLOSSARIES
\usepackage[toc]{glossaries} % acronym will go in main glossary

\renewcommand{\glsnamefont}[1]{\textit{#1}}

\makeglossaries
\newglossaryentry{acoustic networking}{name={acoustic networking},description={a way of connecting underwater vehicles and other nodes wirelessly using sound waves (since light is rapidly attenuated in sea water). See also \url{http://gobysoft.com/doc/acomms}}}
\newglossaryentry{application}{name={application},description={a collection of code that compiles to a single exectuable unit on your operating system. synonymously (and more precise): processes or binaries}}
\newglossaryentry{pubsub}{name={publish/subscribe},description={a method of communication between processes that is roughly analogous to authors and customers of a newspaper or newsletter. Certain people (applications) publish stories (data) that other people (applications) subscribe for and read in the newsletter. Typically applications perform both tasks, subscribing for some data and publishing others. See also \url{http://en.wikipedia.org/wiki/Publish/subscribe}}}
\newglossaryentry{autonomy architecture}{name={autonomy architecture},description={loosely defined, a collection of software applications and libraries that facilitate communications, decision making, timing, and other utilties needed for making robots function. Another common term for this is autonomy ``middleware''}}
\newglossaryentry{daemon}{name={daemon},description={an application on a Linux/UNIX machine that runs continuously in the background. the \texttt{gobyd} is a server and the Goby applications are clients.}}
\newglossaryentry{star topology}{name={star topology},description={all communications pass through a central mediator (in this case, \texttt{gobyd}) and not directly from any Goby application to another}}
\newacronym[description={a language (in the sense of a programming language) that allows querying or accessing data from a database. For example, if I wanted to know the best baseball players in history and I had a database of players' stats, I could write in SQL the following query that would provide the data I need: \texttt{"SELECT * FROM baseball\_players WHERE batting\_average > 0.300 ORDER BY batting\_average DESC"}}]{sql}{SQL}{Structured Query Language}
\newacronym[description={From \cite{protobuf}: ``Protocol buffers are Google's language-neutral, platform-neutral, extensible mechanism for serializing structured data – think XML, but smaller, faster, and simpler. You define how you want your data to be structured once, then you can use special generated source code to easily write and read your structured data to and from a variety of data streams and using a variety of languages – Java, C++, or Python.''}]{protobuf}{protobuf}{Google Protocol Buffers}
\newglossaryentry{base class}{name={base class},plural={base classes},description={also known as subclass or child class}}
\newglossaryentry{derived class}{name={derived class},plural={derived classes},description={also known as superclass or parent class}}
\newglossaryentry{virtual}{name={virtual},description={A member of a \gls{base class} than can be redefined in a \gls{derived class}. See also \url{http://www.cplusplus.com/doc/tutorial/polymorphism/}}}
\newglossaryentry{synchronous}{name={synchronous},description={From \cite{mw-synchronous}: "recurring or operating at exactly the same period."}}
\newglossaryentry{platform}{name={platform},description={Used to refer to a physical robotic entity, such as an AUV, a topside computer on board a ship, or a buoy.}}
\newglossaryentry{asynchronous}{name={asynchronous},description={From \cite{mw-asynchronous}: " of, used in, or being digital communication (as between computers) in which there is no timing requirement for transmission and in which the start of each character is individually signaled by the transmitting device."}}
\newglossaryentry{multicast}{name={multicast},description={A communications scheme where one application sends messages to a group of applications. Multicast is designed such that the sender only sends once and the network topology is responsible for replicating it as necessary. In general, multicast refers to IP (internet protocol) multicast. In Goby, we use encapsulated \gls{pgm}, which provides a reliability layer to UDP multicast.}}
\newacronym[description={A multidiscplinary research group at the Center for Ocean Engineering (Dept. of Mechanical Engineering) at Massachusetts Institute of Technology. LAMSS focuses on collaborative marine robotics for a variety of acoustic and non acoustic sensing tasks. See \url{http://lamss.mit.edu}.}]{lamss}{LAMSS}{Laboratory for Autonomous Marine Sensing Systems}
\newacronym[description={A multicast protocol designed to ensure a level of reliability at the network layer.}]{pgm}{PGM}{Pragmatic General Multicast}


% DOCUMENT
\raggedright

\begin{document}

\begin{center}
\begin{Large}
Goby Underwater Autonomy Project\\
\vspace{0.5em}
\includegraphics[height=3em]{gobysoft_logo_image_only.eps} \\
\vspace{0.5em}
\end{Large}
\begin{LARGE}
User Manual for Version @GOBY_VERSION@.\\
\vspace{0.5em}
\end{LARGE}
Released on @GOBY_VERSION_DATE@.\\
<\url{https://launchpad.net/goby}>



\end{center}
\vspace{0.5em}
\rule{\textwidth}{1pt}

\vspace{0.5em}

\tableofcontents

%\wrappingon
\linenumberfrequency{1}
\bvnumbersoutside
\linenumberfont{\normalfont\tiny}
\chapterstyle{pedersen}
\setsecheadstyle{\Large\raggedright}
\setsubsecheadstyle{\large\raggedright}
\setsubsubsecheadstyle{\itshape\raggedright}
\setparaheadstyle{\itshape\raggedright}
\setsubparaheadstyle{\itshape\raggedright}

\hangsecnum

\chapter{Introduction}

\section{What is Goby?}

The Goby Underwater Autonomy Project is an \gls{autonomy architecture} tailored for marine robotics with a focus on intervehicle communication.

Currently, Goby has three major parts:

\begin{itemize}
\item Goby-Acomms: The Goby Acoustic Communications library (\verb|goby-acomms|) has been provided since Version 1.0. See the Developers' documentation for details on these library and the various modules it contains at \cite{goby-doc}. Users of the MOOS application \verb|pAcommsHandler| should see Chapter \ref{chap:MOOS}.
\item Goby-Util: A utility library that provide functions for dealing with time, type conversion, binary conversions, etc. This library is intended to be small, as Goby makes use of the C++ Standard Library and Boost for most utility tasks.
\item Goby-Core: An \gls{autonomy architecture} that ties together various marshalling schemes (Google Protocol Buffers, MOOS, LCM, etc.) and provides a message passing middleware based on ZeroMQ (for ethernet) and Goby-DCCL (for acoustic communications). Goby-Core will be provided in release version 3.0.
\end{itemize}

\section{Structure of this Manual}
This manual covers the MOOS Applications that use Goby-Acomms release version 2.0. If you are interested in the C++ Goby-Acomms libraries directly, please read the online Developers' documentation at \cite{goby-doc} In fact, you may want to go download and install Goby now before reading further: \url{https://launchpad.net/goby}.

\section{How to get help}
The Goby community is here to support you. This is an open source project so we have limited time and resources, but you will find that many are willing to contribute their help, with the hope that you will do the same as you gain experience. Please consult these resources and people, probably in this order of preference:

\begin{enumerate}
\item This user manual. % TODO(tes) put in link this manual
\item The Wiki: \url{http://gobysoft.com/wiki}.
\item Questions and Answers on Launchpad: \url{https://answers.launchpad.net/goby}.
\item The developers' documentation: \url{http://gobysoft.com/doc}.
\item Email the listserver \href{mailto:goby@mit.edu}{goby@mit.edu}. Please sign up first: \url{http://mailman.mit.edu/mailman/listinfo/goby}.
\item Email the lead developer (T. Schneider): \href{mailto:tes@mit.edu}{tes@mit.edu}.
\end{enumerate}


\chapter{Goby-Acomms}\label{chap:acomms}
\MakeShortVerb{\!} % makes !foo! == !foo!


\section{Problem}
Acoustic communications are highly limited in throughput. Thus, it is unreasonable to expect ``total throughput'' of all communications data. Furthermore, even if total throughput is achievable over time, certain messages have a lower tolerance for delay (e.g. vehicle status) than others (e.g. CTD sample data). 

Also, in order to make the best use of this available bandwidth, messages need to be compacted to a minimal size before sending (effective encoding). To do this, Goby-Acomms provides an interface to the Dynamic Compact Control Language (DCCL\footnote{the name comes from the original CCL written by Roger Stokey for the REMUS AUVs, but with the ability to dynamically reconfigure messages based on mission need. If desired, DCCL can be configured to be backwards compatible with a CCL network using CCL message number 32}) encoder/decoder. 

\section{Dynamic Compact Control Language: DCCL} \label{sec:dccl}

DCCL allows you to take object based ``messages'' (similar to C structs) defined in the Google Protocol Buffers language and extend them to be more strictly bounded. It provides a set of default encoders for these bounded Protocol Buffers messages (now called DCCL messages) to provide a more minimal encoding than the default Protocol Buffers encoding (which is reasonably decent already, but still has too much overhead for extremely slow links). 

\subsection{Configuration: DCCLConfig}

Configuration of individual DCCL messages is done within the .proto definition. All the non-message specific available configuration for !goby::acomms::DCCLCodec! is given in its TextFormat form as:

\boxedverbatiminput{@RELATIVE_CMAKE_CURRENT_SOURCE_DIR@/includes/dccl_config.pb.cfg}
\resetbvlinenumber

\begin{itemize}
\item !crypto_passphrase!: If provided, this preshared key is used to encrypt the body of all messages using AES (Rijndael) encryption. Omit this field to turn off encryption. Note that the contents of messages received by nodes with the wrong encryption key are undefined, and such failure is not currently detected.
\item !id_codec!: The codec used to encode the message identifier (DCCL ID). !VARINT! uses a one (id: 0-127) or two byte (id: 127-32768) encoding on the wire. !LEGACY_CCL! adds another byte to DCCL messages (0x20) to the least significant (first) end to allow interoperability with REMUS CCL. Use !VARINT! unless you need to interoperate with CCL networks since you will save a byte that can be better used elsewhere.
\end{itemize}

\subsection{Configuration: Designing DCCL messages using Protocol Buffers Extensions}

A guide to designing DCCL messages is given at \url{http://gobysoft.com/doc/2.0/acomms_dccl.html} along with a full list of the DCCL extensions to google::protobuf::MessageOptions and google::protobuf::FieldOptions. Therefore, we will not replicate that information here.

\section{Time dependent priority queuing: Queue} \label{sec:queue}

Goby-Queue manages a queue for each DCCL message. When it is prompted by data by the modem, it has a priority "contest" between the queues. the queue with the current highest priority (as determined by the !value_base! and !ttl! fields) is selected. The next message in that queue is then provided to the modem to send. For modem messages with multiple frames per packet, each frame is a separate contest. Thus a single packet may contain frames from different
 queues (e.g. a rate 5 PSK packet has eight 256 byte frames. frame 1 might grab a STATUS message since that has the current highest queue. then frame 2 may grab a BTR message and frames 3-8 are filled up with CTD messages (e.g. STATUS is in blackout, BTR queue is empty)). See \url{http://gobysoft.com/doc/2.0/acomms_queue.html} for more information.

\subsection{Configuration: QueueManagerConfig}

Most of the configuration for queuing messages is done within the .proto message definition of a given DCCL message. The rest of the configuration options for !goby::acomms::QueueManager! are:

\boxedverbatiminput{@RELATIVE_CMAKE_CURRENT_SOURCE_DIR@/includes/queue_config.pb.cfg}
\resetbvlinenumber

\begin{itemize}
\item !modem_id!: A unique integer value for this particular vehicle (like a MAC address). Should be as small as possible for optimal bounding of the source and destination fields of the message. 0 is reserved for broadcast (analogous to 255.255.255.255 for IPv4).
\item !manipulator_entry!: Manipulates the queuing behavior for a given message.
\begin{itemize}
\item !protobuf_name!: String represented the Protobuf message to manipulate. Messages are named the same as !google::protobuf::Descriptor::full_name()!, which is the !package! followed by the message name, separated by dots: e.g. ``goby.acomms.protobuf.ModemTransmission''.
\item !manipulator!: One or more manipulators to apply to the queuing of this message.
\begin{itemize}
\item !NO_MANIP!: A do nothing (nop) manipulator. Same as leaving omitting this field.
\item !NO_QUEUE!: Do not queue this message when generated on this node (but messages will still be received (dequeued).
\item !NO_DEQUEUE!: Do not dequeue (receive) this message on this node (but messages will be queued). When both !NO_QUEUE! and !NO_DEQUEUE! are set, there isn't much point to having the message loaded at all.
\item !LOOPBACK!: Dequeue all instances of this message immediately upon queuing. The message is still queued and sent to its addressed destination. Often used with !PROMISCUOUS!.
\item !ON_DEMAND!: A special (advanced) feature where QueueManager assumes this queue is always full and asks for data immediately from the application upon request from the modem side. Useful for ensuring time sensitive data does not get stale.
\item !LOOPBACK_AS_SENT!: Like loopback, but rather than dequeuing upon queuing, this manipulator dequeues a copy locally upon a data request from the modem. Often used with !PROMISCUOUS!.
\item !PROMISCUOUS!: Dequeue all messages of this type even if this !modem_id! does not match the destination address.
\item !NO_ENCODE!: Same as !NO_QUEUE!, provided for backwards compatibility with Goby v1.
\item !NO_DECODE!: Same as !NO_DEQUEUE!, provided for backwards compatibility with Goby v1.
\end{itemize}
\end{itemize}
\end{itemize}


\section{Time Division Multiple Access (TDMA) Medium Access Control (MAC): AMAC} \label{sec:amac}

The AMAC unit uses time division (TDMA) to attempt to ensure a collision-free acoustic channel.

AMAC supports two variants of the TDMA MAC scheme: centralized and decentralized. As the names suggest, Centralized TDMA (!type: MAC_POLLED!) involves control of the entire cycle from a single master node, whereas each node's respective slot is controlled by that node in Decentralized TDMA. Within decentralized TDMA, Goby supports a fixed (preprogrammed) cycle (!type: MAC_FIXED_DECENTRALIZED!) that can be updated by the application. The autodiscovery mode (!type: MAC_AUTO_DECENTRALIZED!) supported in version 1 is no longer provided in version 2. To disable the AMAC, use (!type: MAC_NONE!). See \url{http://gobysoft.com/doc/2.0/acomms_mac.html} for more details.

\subsection{Configuration: MACConfig}

The !goby::acomms::MACManager! is basically a !std::list<goby::acomms::protobuf::ModemTransmission>!. Thus, its configuration is primarily such an initial list of these ``slot''s. Since !ModemTransmission! is extensible to handle different modem drivers, the AMAC configuration is also automatically extended. Some fields in !ModemTransmission! do not make sense to configure !goby::acomms::MACManager! with, so these are omitted here:

\boxedverbatiminput{@RELATIVE_CMAKE_CURRENT_SOURCE_DIR@/includes/mac_config.pb.cfg}
\resetbvlinenumber


Further details on these configuration fields: 
\begin{itemize}
\item !type!: type of Medium Access Control. See \url{http://gobysoft.com/doc/2.0/acomms_mac.html#amac_schemes} for an explanation of the various MAC schemes.
\item !slot!: use this repeated field to specify a manual polling or fixed TDMA cycle for the  !type: MAC_FIXED_DECENTRALIZED! and  !type: MAC_POLLED!. 
\begin{itemize}
\item !src!: The sending !modem_id! for this slot. Setting both src and dest to 0 causes AMAC to ignore this slot (which can be used to provide a blank slot).
\item !dest!: The receiving !modem_id! for this slot. Omit or set to -1 to allow next datagram to set destination.
\item !rate!: Bit-rate code for this slot (0-5). For the WHOI Micro-Modem 0 is a single 32 byte packet (FSK), 2 is three frames of 64 bytes (PSK), 3 is two frames of 256 bytes (PSK), and 5 is eight frames of 256 bytes (PSK).
\item !type!: Type of transaction to occur in this slot. If !DRIVER_SPECIFIC!, the specific hardware driver governs the type of this slot.
\item !slot_seconds!: The duration of this slot, in seconds.
\item !unique_id!: Integer field that can optionally be used to identify certain types of slots. For example, this allows integration of an in-band (but otherwise unrelated) sonar with the modem MAC cycle.
\end{itemize} 
\end{itemize} 


Relevant extensions of !goby::acomms::protobuf::ModemTransmission! for the WHOI Micro-Modem driver (!DRIVER_WHOI_MICROMODEM!):

\boxedverbatiminput{@RELATIVE_CMAKE_CURRENT_SOURCE_DIR@/includes/mac_mmdriver.pb.cfg}
\resetbvlinenumber

Several examples:
\begin{itemize}
\item Continous uplink from node 2 to node 1 with a 15 second pause between datagrams (node 1's configuration; same for node 2 except for !modem_id = 2!):
\begin{boxedverbatim}
modem_id: 1
type: MAC_FIXED_DECENTRALIZED
slot { src: 2  dest: 1  type: DATA  slot_seconds: 15 }
\end{boxedverbatim}
\resetbvlinenumber
\item Equal sharing for three vehicles (destination governed by next data packet):
\begin{boxedverbatim}
modem_id: 1 # 2 or 3 for other vehicles
type: MAC_FIXED_DECENTRALIZED
slot { src: 1  type: DATA  slot_seconds: 15 }
slot { src: 2  type: DATA  slot_seconds: 15 }
slot { src: 3  type: DATA  slot_seconds: 15 }
\end{boxedverbatim}
\resetbvlinenumber
\item Three vehicles with both data and WHOI Micro-Modem two-way ranging (ping):
\begin{boxedverbatim}
modem_id: 1 # 2 or 3 for other vehicles
type: MAC_FIXED_DECENTRALIZED
slot { src: 1  type: DATA  slot_seconds: 15 }
slot { 
  src: 1
  dest: 2
  type: DRIVER_SPECIFIC 
  [micromodem.protobuf.type]: MICROMODEM_TWO_WAY_PING
  slot_seconds: 5
}
slot { 
  src: 1
  dest: 3
  type: DRIVER_SPECIFIC 
  [micromodem.protobuf.type]: MICROMODEM_TWO_WAY_PING
  slot_seconds: 5
}
slot { src: 2  type: DATA  slot_seconds: 15 }
slot { src: 3  type: DATA  slot_seconds: 15 }
\end{boxedverbatim}
\resetbvlinenumber
\item One vehicle interleaving data and REMUS long-base-line (LBL) navigation pings:
\begin{boxedverbatim}
modem_id: 1
type: MAC_FIXED_DECENTRALIZED
slot { src: 1  type: DATA  slot_seconds: 15 }
slot { 
  src: 1
  dest: 2
  type: DRIVER_SPECIFIC 
  [micromodem.protobuf.type]: MICROMODEM_REMUS_LBL_RANGING
  [micromodem.protobuf.remus_lbl] {
    enable_beacons: 0xf   # enable all four: b1111
    turnaround_ms: 50
    lbl_max_range: 500 # meters
  }
  slot_seconds: 5
}
\end{boxedverbatim}
\resetbvlinenumber
\end{itemize}

\section{Abstract Acoustic (or other slow link) Modem Driver: ModemDriver} \label{sec:driver}

The ModemDriver unit provides a common interface to any modem capable of sending datagrams. It currently supports the WHOI Micro-Modem acoustic modem, UDP over the Internet, and is extensible to other acoustic (or slow link) modems. More details on the ModemDriver are available here: \url{http://gobysoft.com/doc/2.0/acomms_driver.html}.

\subsection{Configuration: DriverConfig}

Base driver configuration:

\boxedverbatiminput{@RELATIVE_CMAKE_CURRENT_SOURCE_DIR@/includes/driver_config.pb.cfg}
\resetbvlinenumber

Extensions for the WHOI Micro-Modem (!DRIVER_WHOI_MICROMODEM!):
\boxedverbatiminput{@RELATIVE_CMAKE_CURRENT_SOURCE_DIR@/includes/driver_mmdriver.pb.cfg}
\resetbvlinenumber

Extensions for the example driver (!DRIVER_ABC_EXAMPLE_MODEM!):
\boxedverbatiminput{@RELATIVE_CMAKE_CURRENT_SOURCE_DIR@/includes/driver_abc_driver.pb.cfg}
\resetbvlinenumber

Extensions for the MOOS uField driver (!DRIVER_UFIELD_SIM_DRIVER!):
\boxedverbatiminput{@RELATIVE_CMAKE_CURRENT_SOURCE_DIR@/includes/driver_ufield.pb.cfg}
\resetbvlinenumber

Extensions for the ZeroMQ/Protobuf storage driver (!DRIVER_PB_STORE_SERVER!):
\boxedverbatiminput{@RELATIVE_CMAKE_CURRENT_SOURCE_DIR@/includes/driver_pb.pb.cfg}
\resetbvlinenumber

Extensions for the UDP driver (!DRIVER_UDP!):
\boxedverbatiminput{@RELATIVE_CMAKE_CURRENT_SOURCE_DIR@/includes/driver_udp.pb.cfg}
\resetbvlinenumber


\DeleteShortVerb{\!}

\chapter{Goby Common Modules}\label{chap:common}
\MakeShortVerb{\!} % makes !foo! == !foo!

\section{Goby Common Applications}\label{sec:base_cfg}

The Goby Common applications use a validating configuration reader based on the Google Protocol Buffers TextFormat class. The configuration of any given application is available by passing the !--example_config! flag (or !-e! for short) to that application. Additionally, any of the configuration that may be given in a file is also available as command line options. Provide !--help! (or !-h!) to see the command line options.

They all share a common subset of the configuration (!base!):

\boxedverbatiminput{@RELATIVE_CMAKE_CURRENT_SOURCE_DIR@/includes/base.pb.cfg}
\resetbvlinenumber

\begin{itemize}
\item !app_name!: Name of the application (defaults to binary name, i.e. oart of argv[0] after last !/!).
\item !loop_freq!: How often to run the synchronous !loop! method. 
\item !platform_name!: Name of the node or platform this is running on .
\item !pubsub_config!: Socket configuration for the publish-subscribe part of Goby-Common. If omitted, no connections or bindings will be made (if an application is standalone).
\begin{itemize}
\item !publish_socket!: The socket used for publishing messages. 
\begin{itemize}
\item !socket_type!: Must always be !PUBLISH! (you can safely omit the field here).
\item !socket_id!: Generally a unique id, unless you want several sockets of the same type to send and receive together. You can safely omit this field; it defaults to 103999.
\item !transport!: !IPC! (UNIX sockets), !TCP!, !PGM! (Pragmatic General Multicast), !EPGM! (PGM encasulated in UDP). In generally, you will use !TCP! or !IPC!.
\item !connect_or_bind!: !CONNECT! is used on the client side, !BIND! is used on the server side. Generally, you will !BIND! the side on a well-known location, and !CONNECT! the sides that may be more dynamic.
\item !ethernet_address!: For !TCP!, !PGM! and !EPGM!, the ethernet address to use.
\item !multicast_address!: For !PGM! and !EPGM!, the multicast address of the group to join.
\item !ethernet_port!: The network port to connect or bind to.
\item !socket_name!: For !IPC!, the name (path) of the UNIX socket to create or connect to.
\end{itemize}
\item !subscribe_socket!: The socket used for received subscribed messages. Except where noted, the fields are the same as for !publish_socket!. 
\begin{itemize}
\item !socket_type!: Must always be !SUBSCRIBE! (you can safely omit the field here).
\item !socket_id!: Generally a unique id, unless you want several sockets of the same type to send and receive together. You can safely omit this field; it defaults to 103998.
\end{itemize}
\end{itemize}
\item !additional_socket_config!: (Advanced) Used to add additional ZeroMQ connections or bindings. 
\item !glog_config!: Configure the !goby::glog! logging utility. 
\begin{itemize}
\item !tty_verbosity!: Verbosity of the debug logging to standard output in the controlling terminal. Choose !DEBUG1!-!DEBUG3! for various levels of debugging output, !VERBOSE! for some text terminal output, !WARN! for warnings only, and !QUIET! for no terminal output.
\item !file_log!: A repeated field to log the debugging output to one or more files. If omitted, no files are logged.
\begin{itemize}
\item !file_name!: Path to file to log. The symbol !%1%! (if present) will be replaced by the current UTC date and time at application launch. %
\item !verbosity!: Verbosity of this file log. Same enumeration options as  !tty_verbosity!.
\end{itemize}
\end{itemize}
\end{itemize}

\section{Liaison}\label{sec:liaison}

Goby Liaison (!goby_liaison!) is an extensible web-browser based GUI for managing various aspects of Goby. It is written using the Wt \cite{wt} library and allows users to manage their Goby systems from any machine (GNU/Linux, Windows, Mac OS X) running a modern web browser (e.g. Firefox, Chrome).

The majority of Liaison is provided by plugin shared libraries that are loaded at runtime using the environmental variable !GOBY_LIAISON_PLUGINS!, which is a colon separated list of libraries (either absolute paths or in paths known to !ld!, such as !/usr/lib!).

The core of Goby Liaison is a server that allows connections from one or more clients through any major modern web browser. The core configuration options are given by:

\boxedverbatiminput{@RELATIVE_CMAKE_CURRENT_SOURCE_DIR@/includes/liaison.pb.cfg}
\resetbvlinenumber

\begin{itemize}
\item !base!: Shared configuration for all !goby_common! applications. See section \ref{sec:base_cfg}.
\item !http_address!: IP address or domain name for the interface to bind on. Use !0.0.0.0! to bind on all interfaces. Use !localhost! to allow connections only from the local machine for security.
\item !http_port!: TCP port to bind on.
\item !docroot!: Path to the Wt !docroot!, where various resources are found (e.g. CSS, images, etc.). The default is usually correct for your installation.
\item !additional_wt_http_params!: Additional command line parameters (separated by spaces) to pass to the Wt server. See \url{http://www.webtoolkit.eu/wt/doc/reference/html/overview.html#config_wthttpd}.
\item !update_freq!: How often to update elements that require data from the server side without client input.
\item !load_shared_library!: Load a shared library (probably containing Google Protobuf messages) for use.
\item !load_proto_file!: Load a !.proto! file directly and compile it at runtime for use. When possible, use !load_shared_library!.
\item !load_proto_dir!: Path to a directory containing !.proto! files. All the !.proto! files in this directory will be loaded and compiled for use.
\item !start_paused!: For modules that require server side updates without client input, setting this true will start up Liaison with these modules paused. This prevents any server side initiated data from being pushed to the client. Set true for use on low-throughput links (e.g. wireless at sea).
\end{itemize}

Additional configuration may be available from the loaded plugins. For example, see the MOOS plugins in section \ref{sec:moos_liaison}.

To connect to a server using the default configuration, simply type !http://localhost:54321! into the address bar of your favorite web browser.

\section{Gateway Applications}

Goby, which uses ZeroMQ as a transport layer, sometimes also needs to talk to other systems using incompatible transport mechanisms. To do this, ``gateway'' applications can be developed that pass packets between the ZeroMQ (Goby) ``world'' and the other system's world. Thus far, one gateway has been written, the !moos_gateway_g! (see section \ref{sec:moos_gateway_g}) for interfacing with the MOOS middleware.

\chapter{Goby MOOS Modules}\label{chap:MOOS}
\MakeShortVerb{\!} % makes !foo! == !foo!

The acoustic communications portion of Goby was developed originally for the MOOS autonomy architecture. Thus, the relevant MOOS modules !pAcommsHandler! and others are still maintained (in goby/src/moos) for the use of the !MOOS-IvP! community. !MOOS-IvP! is explained in \cite{moos-ivp-jfr} and is available at \url{http://moos-ivp.org}. The usage of these modules is documented here. See \url{http://gobysoft.org/wiki/InstallingGoby} for how to install Goby.


\section{Goby MOOS Applications} \label{sec:goby_moos_app}

The Goby MOOS applications share a common subclass of CMOOSApp that provides a validating configuration reader based on the Google Protocol Buffers TextFormat class. The configuration is still embedded within the .moos file, but the syntax is somewhat different. Here you can control logging to a text file and terminal verbosity. You can also initialize a variable in the MOOS database at startup. Many of these parameters will automatically be set to a global MOOS variable (specified outside any ProcessConfig block) if left empty. For example, the global MOOS variable !LatOrigin! will set the Goby MOOS configuration variable !common::lat_origin!. This allows Goby MOOS applications to conform to MOOS \textit{de facto} conventions.

Any Goby MOOS application will give all its valid configuration parameters with \begin{verbatim}
> pGobyApp --example_config
\end{verbatim} 

\boxedverbatiminput{@RELATIVE_CMAKE_CURRENT_SOURCE_DIR@/includes/common.pb.cfg}
\resetbvlinenumber

Some details about the configuration values:

\begin{itemize}
\item !log!: boolean to indicate whether to log terminal output or not to files in the path by !log_path!.
\item !log_path!: folder to log all terminal output to for later debugging. Similar to system logs in /var/log.
\item !log_verbosity!: verbosity of the log file. See !verbosity! for the various settings.
\item !community!: the name of the current vehicle community. If omitted, read from the !Community=! global MOOS configuration field.
\item !lat_origin!: a decimal degrees latitude indicating the local cartesian origin. If omitted, read from the !LatOrigin=! global MOOS configuration field.
\item !lon_origin!: a decimal degrees longitude indicating the local cartesian origin. If omitted, read from the !LongOrigin=! global MOOS configuration field.
\item !app_tick!: same as AppTick.
\item !comm_tick!: same as CommsTick.
\item !verbosity!: choose !DEBUG1!-!DEBUG3! for various levels of debugging output, !VERBOSE! for some text terminal output, !WARN! for warnings only, and !QUIET! for no terminal output.
\item !show_gui!: if true, the running terminal opens an NCurses GUI helpful to debugging and visualizing the many data flows of pAcommsHandler. The verbosity in this GUI is governed by !verbosity!.
\item !initializer!: since many times it is useful to have a MOOS variable including in a message that remains static for a given mission (vehicle name, etc), we give the option to publish initial MOOS variables here (for later use in messages [until overwritten, of course]). If !global_cfg_var! is set, pAcommsHandler looks for a global (i.e. specified at the top of the MOOS file or outside any !ProcessConfig! blocks) value in the .moos file with the name to the right of the colon and publishes it to a MOOS variable with the name to the left of the colon. For example:
\begin{verbatim}
initializer { global_cfg_var: "LatOrigin" moos_var: "LAT_ORIGIN" } 
\end{verbatim}
\resetbvlinenumber
looks for a variable in the .moos file called !LatOrigin! and publishes it to the MOOSDB as a double variable !LAT_ORIGIN! with the value given by !LatOrigin!.
\end{itemize}


\section{pAcommsHandler}
\label{sec:pacommshandler} 

pAcommsHandler provides a:
\begin{enumerate}
\item MOOS Application wrapper for the Goby-Acomms communication library.
\item set of translation tools for converting the DCCL messages (written as an extension of Google Protocol Buffers) to MOOS types (strings and doubles) and vice-versa.
\item full backwards-compatibility support module for version 1 XML messages. 
\end{enumerate}

This section describes only the parts relevant for interface to MOOS (variables and translator entries that allow you to read and write to and from DCCL (Protobuf) messages). You should read Chapter \ref{chap:acomms} before starting this section and reference it as necessary.

\subsection{Parameters for the pAcommsHandler Configuration Block}\label{sec:pAcommsHandler:config}

\subsubsection{Example moos file}

pAcommsHandler has a large number of configuration options, many of which you will never use or leave as default. You can always get a complete listing of MOOS file parameters with their syntax by running
\begin{verbatim}
> pAcommsHandler --example_config
\end{verbatim}
\resetbvlinenumber

These configuration values are provided here (with $\ldots$ where the relevant configuration is provided elsewhere in this document):

\boxedverbatiminput{@RELATIVE_CMAKE_CURRENT_SOURCE_DIR@/includes/pAcommsHandler_reduced.moos}
\resetbvlinenumber


\subsubsection{Filling out the .moos file}\label{sec:pAcommsHandler_moos_file}

Many of the parameters are sufficiently explained in the above list of configuration parameters. What follows is a detailed explanation of the parameters that need further explanation.

\begin{itemize}
\item !common!: Parameters that can be set for any of the Goby MOOS applications. See section \ref{sec:goby_moos_app}.
\item !modem_id!: integer that specifies the !modem_id! of this current vehicle / community. For the WHOI Micro-Modem this is the Micro-Modem ``SRC'' configuration parameter (as set by !$CCCFG,SRC,#!). For the remainder of the document, !modem_id! refers to the value !$CCCFG,SRC,modem_id!. This configuration parameter will be set on startup. Setting this within the main block for pAcommsHandler sets it for all the modules (!driver_cfg!, !queue_cfg!, !mac_cfg!) 
\item !driver_type!: 
\begin{itemize}
\item !DRIVER_WHOI_MICROMODEM! is a driver for the WHOI Micro-Modem. 
\item !DRIVER_ABC_EXAMPLE_MODEM! is a simple test ``modem''. Do not use this for real work, but rather for learning how to write new drivers for Goby.
\item !DRIVER_UFIELD_SIM_DRIVER! is a driver for the MOOS-IvP uField toolbox.
\item !DRIVER_PB_STORE_SERVER! is a ZeroMQ (TCP, UNIX sockets) driver for the !goby_store_server! database.
\item !DRIVER_UDP! is a user datagram protocol (UDP) driver. This is probably the easiest driver to start with for learning pAcommsHandler.
\item !DRIVER_NONE! disables the modem driver.
\end{itemize}
\item !driver_cfg!: Configures the base driver and the specific driver selected. See section \ref{sec:driver}.
\item !mac_cfg!: Configures the acoustic Medium Access Control. See section \ref{sec:amac}.
\item !queue_cfg!: Configures the Priority Queuing layer. See section \ref{sec:queue}.
\item !dccl_cfg!: Configures the Dynamic Compact Control Language. See section \ref{sec:dccl}.
\item !route_cfg!: Configures a basic static routing module. This is experimental and subject to change.
\item !moos_var!: Rename any or all of the MOOS variables published by pAcommsHandler.
\item !load_shared_library!: List of paths to shared libraries containing compiled DCCL (Google Protocol Buffers) messages.
\item !load_proto_file!: List of paths to .proto files containing compiled DCCL (Google Protocol Buffers) messages. These will be compiled at runtime and loaded. It is preferable to use !load_shared_library! when possible, as syntactical and type mistakes in the DCCL messages will be caught at compile-time rather than delayed to runtime.
\item !translator_entry!: List of entries indicating when to make (\textit{trigger}) and how to \textit{create} outgoing DCCL messages, and how to \textit{publish} incoming DCCL messages. This can be thought of as providing a generic interface between MOOS strings and Google Protocol Buffers messages.
\item !multiplex_create_moos_var!: Used by !goby_liaison! to publish multiple commands (outgoing messages) on a single MOOS variable.
\item !modem_id_lookup_path!: path to a text file giving the mapping between !modem_id! and vehicle name and type for a given experiment. This file should look like:
\begin{boxedverbatim}
// modem id, vehicle name (should be community name), vehicle type
0, broadcast, broadcast
1, endeavor, ship
3, unicorn, auv
4, macrura, auv
\end{boxedverbatim}
\resetbvlinenumber
\item !transitional_cfg!: Provides the same functionality as !dccl_cfg! does in pAcommsHandler from version 1 of Goby. Behind the scenes, XML messages are read, translated to version 2 Protobuf DCCL messages, and written to the !generated_proto_dir!, and subsequently loaded using !load_proto_file!. The appropriate !translator_entry!s are also created from these messages. Do not use this configuration or the XML representation of DCCL messages for any new projects. See the version 1 documentation (\url{http://gobysoft.org/doc/1.1/}) for more details on the XML representation of DCCL messages.
\end{itemize}

%\subsection{MOOS variables subscribed to by pAcommsHandler}
%
%Except for the user-configured publishes (!translator_entry!), pAcommsHandler uses the \href{http://code.google.com/apis/protocolbuffers/docs/reference/cpp/google.protobuf.text_format.html}{Google Protocol Buffers TextFormat} class for serializing to and parsing from MOOS strings (same as !TECHNIQUE_PREFIXED_PROTOBUF_TEXT_FORMAT!). This saves significant effort in manually parsing strings. You should use these same facilities for creating and reading messages. Two helper functions are provided in \\ \href{http://gobysoft.com/doc/moos__protobuf__helpers_8h.html}{goby/moos/libmoos\_util/moos\_protobuf\_helpers} will help you serialize and parse these messages. See \url{http://gobysoft.com/doc/2.0/acomms.html#protobuf} for a brief overview of Google Protocol Buffers as used in Goby.
%
%\begin{itemize}
%\item !DCCL!: Most variables subscribed to by pAcommsHandler are configured in the message XML files and are designated by the tags \xmltag{src\_var} (used to fetch data for a particular !message_var! within a DCCL message) and \xmltag{trigger\_var} (used to trigger the creatinon of a particular DCCL message and possibly provide some data for that message. See \ref{sec:dccl_overview} for details on the XML configuration. 
%\item !Queue!:
%\begin{itemize}
%\item Subscribes to the variables given in !queue_cfg.queue.in_pubsub_var! for CCL queue sending. The contents of this MOOS variable should be a serialized \href{http://gobysoft.com/doc/modem__message_8proto_source.html}{ModemDataTransmission}). 
%\item !ACOMMS_RANGE_COMMAND! (type: \href{http://gobysoft.com/doc/modem__message_8proto_source.html}{ModemRangingRequest}): You write this to initiate a ranging request outside the MAC schedule. Note in general it is preferable to use the MAC cycle to coordinate data and ranging.
%\end{itemize}
%\item !MAC!: !ACOMMS_MAC_CYCLE_UPDATE! (type: \href{http://gobysoft.com/doc/amac_8proto_source.html}{MACUpdate}) You write this to update the MAC cycle for !MAC_FIXED_DECENTRALIZED! and !MAC_POLLED! modes of operation.
%\end{itemize}
%
%For example, to publish a !ACOMMS_MAC_CYCLE_UPDATE!, you would use code like this:
%\begin{boxedverbatim}
%// provides serialize_for_moos
%#include <goby/moos/libmoos_util/moos_protobuf_helpers.h>
%// provides goby::acomms::protobuf::MACUpdate
%#include <goby/common/amac.pb.h>
%
%...
%
%MyMOOSApp::Iterate()
%{
%  if(do_update_mac)
%  { 
%    using namespace goby::acomms::protobuf;
%    MACUpdate mac_update;
%    mac_update.set_dest(1); // update for us if modem_id == 1
%    // add slot to end of existing cycle
%    mac_update.set_update_type(MACUpdate::ADD);
%    Slot* new_slot = mac_update.add_slot();
%    new_slot->set_src(1);  // send from us
%    new_slot->set_dest(3); // send to vehicle 3
%    new_slot->set_rate(0);
%    new_slot->set_slot_seconds(15);
%    new_slot->set_type(SLOT_DATA);
%    
%    std::string serialized;
%    serialize_for_moos (&serialized, mac_update);
%    m_Comms.Notify("ACOMMS_MAC_CYCLE_UPDATE", serialized);
%  }
%}
%\end{boxedverbatim}
%\resetbvlinenumber
%
%\subsection{MOOS variables published by pAcommsHandler}
%
%Except for DCCL \xmltag{publish\_var}s (which use a printf style syntax), pAcommsHandler uses the Google Protocol Buffers TextFormat class for serializing to MOOS strings. 
%
%\begin{itemize}
%\item !DCCL!: Most variables published by pAcommsHandler are configured in the message XML files and are designated by the tags \xmltag{publish\_var} within a \xmltag{publish} block. See \ref{sec:dccl_overview} for details on the XML configuration. 
%\item !Queue!:
%\begin{itemize}
%\item !ACOMMS_INCOMING_DATA! (type: \href{http://gobysoft.com/doc/modem__message_8proto_source.html}{ModemDataTransmission}) written for all received messages containing a data payload
%\item !ACOMMS_OUTGOING_DATA! (type: \href{http://gobysoft.com/doc/modem__message_8proto_source.html}{ModemDataTransmission}) written for all queued messages containing a data payload
%\item !ACOMMS_RANGE_RESPONSE! (type: \href{http://gobysoft.com/doc/modem__message_8proto_source.html}{ModemRangingReply}) written in response to ranging request (to another modem or LBL beacons)
%\item !ACOMMS_ACK! (type: \href{http://gobysoft.com/doc/modem__message_8proto_source.html}{ModemDataAck}) written when received data is acknowledged acoustically by a third party. Contains the original message.
%\item !ACOMMS_EXPIRE! (type: \href{http://gobysoft.com/doc/modem__message_8proto_source.html}{ModemDataExpire}) written when a message expires (time-to-live [ttl] exceeded) from the queue before being sent (ack = false) or acknowledged (ack = true)
%\item !ACOMMS_QSIZE! (type: \href{http://gobysoft.com/doc/queue_8proto_source.html}{QueueSize}) written when a queue changes size (pop or push) with the new size of the queue.
%\end{itemize}
%\item !MAC!: Does not publish anything.
%\item !ModemDriver!: 
%\begin{itemize}
%\item !ACOMMS_NMEA_IN! (type: string), ModemMsgBase::raw() for all incoming messages ("\$CA..." for WHOI Micro-Modem)
%\item !ACOMMS_NMEA_OUT! (type: string), ModemMsgBase::raw() for all outgoing messages ("\$CC..." for WHOI Micro-Modem)
%\end{itemize}
%\end{itemize}
%
%For example, to read an !ACOMMS_RANGE_RESPONSE!, you would use code like this:
%\begin{boxedverbatim}
%// provides parse_for_moos
%#include <goby/moos/libmoos_util/moos_protobuf_helpers.h>
%// provides goby::acomms::protobuf::ModemRangeReply
%#include <goby/common/modem_message.pb.h>
%
%...
%
%MyMOOSApp::OnNewMail()
%{
%  ...
%  if(moos_msg.GetKey() == "ACOMMS_RANGE_RESPONSE")
%  {
%    using namespace goby::acomms::protobuf;
%    ModemRangeReply range_response;
%    parse_for_moos (serialized, &range_response);
%    
%    // now do what you want to with the nice `range_response` object
%    std::cout << "one way travel time to " << range_response.base().dest() 
%              << " is " << range_response.one_way_travel_time(0) << std::endl;
%  }
%}
%\end{boxedverbatim}
%\resetbvlinenumber
%
%\subsection{Simple complete example MOOS files}
%
%\subsubsection{Example 1: Basic CCL (goby/share/cfg/MOOS/basic\_ccl)}\label{sec:moos_example_1}
%This example sends the bytes !0x020304! from node 1 (!mm1!) to node 2 (!mm2!). It shows use of all the parts of pAcommsHandler except the DCCL encoding / decoding unit. I use !iModemSim! here to simulate the WHOI Micro-Modem. This process is available in moos-ivp-local (\url{http://oceanai.mit.edu/moos-ivp/pmwiki/pmwiki.php?n=Support.Milocal}). You can also easily substitute real modems by removing iModemSim references and changing the !serial_port!.
%
%\paragraph{MOOS file for Node 1: goby/share/cfg/MOOS/basic\_ccl/mm1.moos}
%\boxedverbatiminput{"@RELATIVE_CMAKE_SOURCE_DIR@/share/cfg/MOOS/basic_ccl/mm1.moos"}
%\resetbvlinenumber
%
%\paragraph{MOOS file for Node 2: goby/share/cfg/MOOS/basic\_ccl/mm2.moos}
%\boxedverbatiminput{"@RELATIVE_CMAKE_SOURCE_DIR@/share/cfg/MOOS/basic_ccl/mm2.moos"}
%\resetbvlinenumber
%
%\subsubsection{Example 2: DCCL and CCL (goby/share/cfg/MOOS/ccl\_and\_dccl)}\label{sec:ccl_dccl_example}
%This example sends the DCCL ``Simple Status'' messsage from node 1 (!mm1!) to node 2 (!mm2!). !mm2! sends the REMUS CCL State message to !mm1!. It thus uses all the components of pAcommsHandler. As in the previous example, you can use real modems by removing iModemSim and changing the !serial_port! to the proper real serial port.
%
%\paragraph{MOOS file for Node 1: goby/share/cfg/MOOS/ccl\_and\_dccl/mm1.moos}
%\boxedverbatiminput{"@RELATIVE_CMAKE_SOURCE_DIR@/share/cfg/MOOS/ccl_and_dccl/mm1.moos"}
%\resetbvlinenumber
%
%\paragraph{MOOS file for Node 2: goby/share/cfg/MOOS/ccl\_and\_dccl/mm2.moos}
%\boxedverbatiminput{"@RELATIVE_CMAKE_SOURCE_DIR@/share/cfg/MOOS/ccl_and_dccl/mm2.moos"}
%\resetbvlinenumber
%
%\paragraph{XML definition of Simple Status: goby/xml/simple\_status.xml}
%\boxedverbatiminput{"@RELATIVE_CMAKE_SOURCE_DIR@/share/xml/simple_status.xml"}
%\resetbvlinenumber
%
%\paragraph{Modem Lookup Table: goby/share/cfg/MOOS/ccl\_and\_dccl/modemidlookup.txt}
%\boxedverbatiminput{"@RELATIVE_CMAKE_SOURCE_DIR@/share/cfg/MOOS/ccl_and_dccl/modemidlookup.txt"}
%\resetbvlinenumber


\section{Migrating from Version 1 to Version 2} \label{sec:gobyv1_migrate}


\section{iCommander}\label{sec:icommander} 

\textit{Deprecated. Use goby\_liaison as a replacement.}

\section{pREMUSCodec}

\textit{Deprecated, see section \ref{sec:ccl_dccl_example} for an example of using pAcommsHandler with CCL.}

\DeleteShortVerb{\!}

\chapter{What's next}

That's all for \verb|goby| in Release 2.0. There's still a lot to do so keep tuned. If you want the bleeding edge, you can check out the Goby 3.0 branch with 
\verb|bzr checkout lp:goby/3.0|.

Here's what's on the horizon:
\begin{itemize}
\item Goby-Common: a general purpose interprocess and interplatform communication based on messaging schemes drawn both from the existing marine robotics and global open source communities. The focus is on a high degree of runtime reliability and collaboration between development communities. For advanced users, it provides a transport layer built on ZeroMQ (which supports 20+ languages including C, C++, Java, .NET, Python, and major platforms) for communicating over reliable multicast (PGM) using one or more existing (e.g. MOOS, LCM, Protobuf, CCL, DCCL, ...) messaging schemes. Goby does not mandate a programming language, a messaging scheme, or a development system and thus intends to tie together groups with different goals, styles, and rules. Furthermore, Gateways can be written to interface the ZeroMQ based Goby transport with the native transport systems used by other architectures (e.g. MOOSDB, LCM multicast).
\item Goby-PB: The Google Protocol Buffers / C++ implementation of Goby-Common. For introductory users, it provides an "template" application in C++ that allows object-based messaging (based on Google Protocol Buffers) between processes and platforms without any concern for serialization, routing, sockets, and so on.
\end{itemize}

Stay tuned at \url{https://launchpad.net/goby}. Thanks.



\printglossaries

\bibliographystyle{IEEEtran}
\bibliography{user_manual}
\end{document}

